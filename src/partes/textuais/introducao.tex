\chapter{Introdução}
% ----------------------------------------------------------

Em 2008, Satoshi Nakamoto descreveu em um trabalho que foi chamado de \textit{Bitcoin: A Peer-to-Peer Eletronic Cash System} uma maneira de realizar pagamentos digitais, ao utilizar criptomoedas, de forma descentralizada. \cite{bitcoin} A tecnologia principal que move esse sistema de pagamento é o \textit{Blockchain}, uma maneira de armazenar dados de forma extremamente segura, onde os dados são imutáveis e o sistema pode ser totalmente descentralizado.\cite{blockchain}

A partir dessa tecnologia, em 2014, Vitalik Buterin publicou um trabalho para apresentar o \textit{Ethereum}, um protocolo que roda com a tecnologia de Blockchain e permite criar aplicações decentralizadas através de \textit{Smart Contracts}.\cite{ethereum} \cite{ethereum_yellow} Os \textit{Smart Contracts} é uma forma de escrever um contrato que pode ser executado de forma automática.\cite{smart_contract} \cite{smart_contract_blockchain}

Em um mundo onde tudo está cada vez mais conectado, a necessidade de ter sistemas distribuídos está se mostrando uma peça fundamental para a liberdade individual, de forma que, você possa comprar ou ter algo sem que ninguém possa censurar ou retirar de você.\cite{decentralization}

E a partir disso, é necessário uma forma de você poder realizar serviços para outras pessoas e receber por esse serviço, e nosso trabalho tem como alvo resolver esse problema. Ao pensar sobre as soluções existentes hoje, temos o Workana e Upwork, que são plataformas de \textit{Freelance} que tem toda a sua estrutura centralizada e não realizam pagamentos em criptomoeda.

Com isso, a solução proposta nesse trabalho é a criação de uma plataforma de \textit{Freelance} totalmente descentralizada, de forma que, você possa postar projetos e ser contratado recebendo por esse trabalho em criptomoeda. Além disso, com um sistema de resolução de conflitos e outro para pontuação, a ideia é incentivar bons \textit{Freelancers} a entrarem e continuarem na plataforma.

Um ponto importante de uma plataforma de \textit{Freelance} é a resolução de conflitos e quando será feito os pagamentos, visto que, é possível que haja pessoas má intencionadas querendo roubar o dinheiro de um contratante, ou mesmo, o contratante não querendo pagar o \textit{Freelance}, e a solução para os dois problemas será descrita nos próximos capítulos.

Para construir a plataforma, será utilizado principalmente \textit{Smart Contracts}, que rodará em uma rede de \textit{Blockchain} chamada \textit{Polygon}, uma alternativa para a \textit{Ethereum} que possui taxas de transações muito mais baixas, além de processar as transações mais rapidamente. \cite{polygon}

Além dos \textit{Smart Contracts}, será criado um site que posteriormente será disponibilizado aos usuários através do protocolo \textit{IPFS}, que é um sistema de armazenamento permanente de arquivos de forma descentralizada. Uma vez que os arquivos estejam no sistema do \textit{IPFS}, para ser excluído, todos na rede que possuem uma cópia precisam ativamente excluir essa cópia.\cite{ipfs}

Para esse trabalho, é esperado no final dele a criação dos \textit{Smart Contracts}, assim como, um site hospedado no protocolo \textit{IPFS} para servir de interface para os \textit{Smart Contracts} que estarão disponibilizados na rede da \textit{Polygon}. Contudo, esse trabalho não irá abranger a comunicação entre duas pessoas de forma descentralizada e anônima, a comunicação ainda será responsabilidade do contratante e do \textit{Freelancer}.

Este trabalho está organizado da seguinte maneira: no Capítulo 2 será feito uma revisão teórica sobre os conceitos apresentados nessa introdução para explicar sobre a descentralização e como é feita a segurança de todo o sistema. No Capítulo 3 será apresentado as tecnologias e serviços utilizados para a criação da plataforma. No Capítulo 4 será mostrado detalhes da implementação da plataforma, além de mostrar o design dos \textit{Smart Contracts} e da estrutura do site construido, assim como, também será descrito as dificuldades e limitações de um sistema descentralizado descritas por Chu (\citeyear{curses_of_blockchain}). Por fim, no Capítulo 5 será apresentado os resultados, e no Capítulo 6, as conclusões assim como sugestões para trabalhos futuros.
