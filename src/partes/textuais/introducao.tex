\chapter{Introdução}
% ----------------------------------------------------------

Em 2008, Satoshi Nakamoto descreveu em um trabalho que foi chamado de \textit{Bitcoin: A Peer-to-Peer Eletronic Cash System} uma maneira de realizar pagamentos digitais, ao utilizar criptomoedas, de forma descentralizada. Isso significa que para realizar a transferência de um valor para outra pessoa, não é necessário que uma instituição financeira para validar que quem está transferindo realmente possui o valor, a própria rede do \textit{Bitcoin} garantirá essa validação, resolvendo assim o problema do gasto-duplo.\cite{bitcoin}

A tecnologia principal que move esse sistema de pagamento é o \textit{Blockchain}, uma maneira de armazenar dados de forma extremamente segura que funciona como uma lista de blocos que são interligados uns aos outros de forma a garantir que os dados registrados nos blocos sejam imutáveis.\cite{blockchain}

A partir dessa tecnologia, em 2014, Vitalik Buterin publicou um trabalho para apresentar o \textit{Ethereum}, um protocolo que roda com a tecnologia de Blockchain e permite criar aplicações decentralizadas através de \textit{Smart Contracts}.\cite{ethereum} \cite{ethereum_yellow} Os \textit{Smart Contracts} é uma forma de escreve um contrato que pode ser executado de forma automática.\cite{smart_contract} \cite{smart_contract_blockchain}

Com essas tecnologias, em 2022, é possível realizar pagamentos entre pessoas de qualquer parte do mundo sem que ninguém possa impedir, e também ter a posse digital de uma obra de arte onde você tem uma prova imutável de que você é o dono dela. Em um mundo onde tudo está cada vez mais conectado, a necessidade de ter sistemas distribuídos está se mostrando uma peça fundamental para a liberdade individual, de forma que, você possa comprar ou ter algo sem que ninguém possa censurar ou retirar de você.\cite{decentralization}

Contudo, ainda é necessário que você possua dinheiro no mundo digital para que você possa realizar pagamentos e compras, e uma das formas de ganhar dinheiro é através de \textit{Freelance}, onde você realiza algum trabalho e recebe por esse trabalho. Ao pensar sobre as soluções existentes hoje, temos o Workana e Upwork mas são plataformas que tem toda a sua estrutura centralizada e não realizam pagamentos em criptomoeda.

Com isso, o problema que esse trabalho se propõe a resolver é a criação de uma plataforma de \textit{Freelance} totalmente descentralizada, de forma que, você possa postar projetos e ser contratado para um projeto recebendo por esse trabalho em criptomoeda. Além disso, com um sistema de resolução de conflitos e outro para pontuação, a ideia é incentivar bons \textit{Freelancers} a entrarem e continuarem na plataforma.

Essa plataforma tem como o objetivo de ser simples de ser usada, totalmente descentralizada com sistema de pontuação para premiar e evidenciar bons \textit{Freelancers}. Além disso, ao ser totalmente descentralizado, ela se diferenciará de outras plataformas de \textit{Freelance} em \textit{Blockchain} que possuem alguns dos seus sistemas centralizados em troca de melhorar a usabilidade.

Um ponto importante de uma plataforma de \textit{Freelance} é a resolução de conflitos e quando será feito os pagamentos, visto que, é possível que haja pessoas má intencionadas querendo roubar o dinheiro de um contratante, ou mesmo, o contratante não querendo pagar o \textit{Freelance}, esses dois problemas serão descritos nos próximos capítulos.

Para construir a plataforma, será utilizado principalmente \textit{Smart Contracts}, que rodará em uma rede de \textit{Blockchain} chamada \textit{Polygon}, uma alternativa para a \textit{Ethereum} que possui taxas de transações muito mais baixas, além de processar as transações mais rapidamente. E assim como na \textit{Ethereum}, que possui a moeda principal \textit{ether}, na \textit{Polygon} temos o Matic, que será usado para realizar o pagamento das taxas das transações.\cite{polygon}

Além dos \textit{Smart Contracts}, será criado um site que posteriormente será disponibilizado aos usuários através do protocolo \textit{IPFS}, que é um sistema de armazenamento permanente de arquivos de forma descentralizada. Uma vez que os arquivos estejam no sistema do \textit{IPFS}, para ser excluído, todos na rede que possuem uma cópia precisam ativamente excluir essa cópia.\cite{ipfs}

Para esse trabalho, é esperado no final dele a criação dos contratos inteligentes, assim como, um site para servir de interface para os contratos que estarão disponibilizados em uma rede de \textit{Blockchain} e no protocolo \textit{IPFS}. Contudo, esse trabalho não irá abranger a comunicação entre duas pessoas de forma descentralizada e anônima, a comunicação ainda será responsabilidade do contratante e do \textit{Freelancer}.

Este trabalho está organizado da seguinte maneira: no Capítulo 2 é será feito uma revisão teórica sobre os conceitos apresentados nessa introdução para explicar sobre a descentralização e como é feita a segurança de todo o sistema. No Capítulo 3 será apresentado as tecnologias e serviços utilizados para a criação da plataforma. No Capítulo 4 será mostrado detalhes da implementação da plataforma, além de mostrar o design dos \textit{Smart Contracts} e da estrutura do site construido assim como também será descrito as dificuldades e limitações de um sistema descentralizado. Por fim, nos Capítulos 5 será apresentado os conceitos, e no Capítulo 6, as conclusões assim como sugestões para trabalhos futuros.\cite{curses_of_blockchain}
